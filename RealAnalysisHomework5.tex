\documentclass[11pt]{article}

\usepackage{amsmath,amsthm}
\usepackage{amsfonts}
\usepackage{amssymb}
\usepackage{enumitem}
\usepackage{mathtools}

\DeclarePairedDelimiter{\abs}{\lvert}{\rvert}
\DeclarePairedDelimiter{\norm}{\lVert}{\rVert}

\DeclarePairedDelimiter{\paren}{\lparen}{\rparen}
\DeclarePairedDelimiter{\bracks}{\lbrack}{\rbrack}
\DeclarePairedDelimiter{\braces}{\lbrace}{\rbrace}

\title{Real Analysis - Assignment 5}
\author{Paul Wintz}
\date{February 21, 2018}

\begin{document}
\maketitle

%Additional problems not to be turned in: 4.2.1, 4.2.2, 4.3.5

\section*{4.2 Points}
\paragraph{Exercise 4.2.4.} \textit{Show (1) that no interior point of a set can be a boundary point, (2) that it is possible for an accumulation point to be a boundary point, and (3) that every isolated point must be a boundary point.} \newline

(1) If a point P is an interior point for set S, then there is an interval around P that is within S. If there is an interval that is completely with within S, then there exists an interval that contains P such that it does not contain any points that are not in S, therefore P cannot be a boundary point. %TODO: Details

(2) Say that P is a point such that for all 0<delta<M, the interval (p-delta, p), contains an infinite number of points in the set S and the interval (p, p+delta) contains no points in S. Clearly, by definition, P is a boundary point of S. Also by definition, P must be a point of accumulation of S because there are infinite points in (p-delta, p+delta) that are in S. Therefore it is possible for a point ot be both a point of accumulation and a boundary point.

(3) 

\paragraph{Exercise 4.2.5.} Let $E$ be a nonempty set of real numbers that is bounded above but has no maximum.
Let $x = \sup E$.
Show that $x$ is a point of accumulation of $E$.
Is it possible for $x$ to also be an interior point of $E$?
Is $x$ a boundary point of $E$?

\paragraph{Exercise 4.2.16.} Show that there is no set which has the set $\mathbb{Q}$ as its set of accumulation points.

\paragraph{Exercise 4.2.21.} Let $E$ be a set and $\{x_n\}$ a sequence of points, not necessarily elements of $E$.
Suppose that $\lim_{x\rightarrow\infty} x_n = x$ and that $x$ is an interior point of $E$.
Show that there is an integer $N$ so that $x_n \in E$ for all $n > N$.

\section*{4.3 Sets}
\paragraph{Exercise 4.3.9.} Show that the set $E'$ of points of accumulation of any set $E$ must be closed.

\paragraph{Exercise 4.3.13.} If $A$ is open and $B$ is closed, what can you say about the sets $A \setminus B$ and $B \setminus A$?

\paragraph{Exercise 4.3.18.} Let $\{x_n\}$ be a sequence of real numbers.
Let $E$ denote the set of all numbers $z$ that have the property that there exists a subsequence $\{x_{n_k}\}$ convergent to $z$.
Show that $E$ is closed.

\end{document}
