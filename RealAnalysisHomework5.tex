\documentclass[11pt]{article}

\usepackage{amsmath,amsthm}
\usepackage{amsfonts}
\usepackage{amssymb}
\usepackage{enumitem}
\usepackage{mathtools}

\DeclarePairedDelimiter{\abs}{\lvert}{\rvert}
\DeclarePairedDelimiter{\norm}{\lVert}{\rVert}

\DeclarePairedDelimiter{\paren}{\lparen}{\rparen}
\DeclarePairedDelimiter{\bracks}{\lbrack}{\rbrack}
\DeclarePairedDelimiter{\braces}{\lbrace}{\rbrace}

\title{Real Analysis - Assignment 5}
\author{Paul Wintz}
\date{February 21, 2018}

\newcommand{\limittoinfy}[1]{\lim_{#1 \to \infty}}
\def \limittoinfyn{\limittoinfy{n}}

\newcommand{\sumto}[1]{\sum_{n=1}^{#1}}
\def \sumtoinfy {\sum_{n=1}^{\infty}}

\newcommand{\set}[1]{\{#1\}}
\def \sn {s_n}
\def \tn {t_n}
\def \xn {x_n}
\def \an {a_n}
\def \bn {b_n}


\begin{document}
\maketitle

%Additional problems not to be turned in: 4.2.1, 4.2.2, 4.3.5

\paragraph{Exercise 4.2.4.} \textit{Show (1) that no interior point of a set can be a boundary point, (2) that it is possible for an accumulation point to be a boundary point, and (3) that every isolated point must be a boundary point.} \newline

(1) If $x$ is an interior point for set $S$, then there is an interval $I = (x - c, x + c) \subseteq S$, for some $c > 0$. Since the interval $I$ is does not contain any points that are not in S, then $x$, by definition, cannot be a boundary point. \newline

(2) Let $x$ be a point such that $\forall 0 < \delta$ the interval $(x - \delta, x)$, contains an infinite number of points in a set $S$ and the interval $(x, x + \delta)$ contains no points in $S$. By definition, $x$ is a boundary point of $S$. And, because there are infinite points in $(x - \delta, x + \delta)$ that are in $S$, $x$ is also be an accumulation point of $S$. Therefore it is possible for a point to be both a point of accumulation and a boundary point. \newline

(3) If $x$ is any isolated point, then there exists a value $\delta > 0$ such that $(x-\delta, x+\delta)$ contains no other points in $S$ aside from $x$. This interval contains many points that are not in $S$, and one point, $x$, that is in $S$. Therefore every isolated point is a boundary point.

\paragraph{Exercise 4.2.5.} \textit{Let $E$ be a nonempty set of real numbers that is bounded above but has no maximum. Let $x = \sup E$. Show that $x$ is a point of accumulation of $E$.}
	
\begin{proof} Because $x$ is the supremum of $E$, we know that $\forall \alpha < x$ ($\alpha$ not necessarily in $E$), there is a value $e \in E$ such that $\alpha < e$. Furthermore, $x$ is not in $E$, otherwise it would be the maximum, so $\forall e \in E$, $e < x$. It then follows directly that $\forall c > 0$, there exists an element $e \in E$ such that $x - c < e < x$. For any value $e_1$ in the interval $(x - c_1, x)$, we can choose a smaller interval, $(x - c_2, x)$, where $c_2 < x - e_1$, that does not contain $e_1$. But, as we showed earlier, this interval must contain an element $e_2 \in E$. By induction, there are an infinite number of elements of $E$ in the interval $(x - c,  x)$, and thus in $(x - c,  x + c)$, therefore $x$ is a point of accumulation.
\end{proof}
		
\paragraph*{} \textit{Is it possible for $x$ to also be an interior point of $E$?} \newline

No. \begin{proof}
The supremum of a set is in the set if and only if it is the maximum of the set. Since $E$ has no maximum, then $x$ is not in the set. Therefore, $\forall c > 0$ the interval $(x - c, x + c)$ will contain a point, $x$, that is not in $E$, so by definition $x$ is not an interior point. 
\end{proof}

\paragraph*{} \textit{Is $x$ a boundary point of $E$?} \newline

Yes
\begin{proof} 
There are no elements in $E$ greater than $x$ in $E$, since $x = \sup E$. But, as we showed earlier, $\forall c > 0$, the interval $I = (x - c,  x + c)$ contains an element of $E$.  Therefore, $I$ contains a point in $E$ and a point not in $E$, so $x$ is a boundary point of $E$.
\end{proof}

\paragraph{Exercise 4.2.16.} \textit{Show that there is no set which has the set $\mathbb{Q}$ as its set of accumulation points.} 

\begin{proof}
We show below, in Exercise 4.3.9, that for any set of real numbers $E$, the set $E'$ of accumulation points of $E$ is closed. We know irrational numbers are points of accumulation for $\mathbb{Q}$, so $\mathbb{Q}$ is not closed, therefore $\mathbb{Q}$ is not the set of accumulation points of any set.
\end{proof}

\newpage
\paragraph{Exercise 4.2.21.} \textit{Let $E$ be a set and $\{x_n\}$ a sequence of points, not necessarily elements of $E$. Suppose that $\lim_{x\rightarrow\infty} x_n = x$ and that $x$ is an interior point of $E$. Show that there is an integer $N$ so that $x_n \in E$ for all $n > N$.} 

\begin{proof}
	By the definition of limits, for all $\epsilon > 0$, there exists a $N$ such that $\forall n > N$, $|x_n - x| < \epsilon$. 
	
	If we expand the absolute value, we get 
	\begin{alignat*}{3}
		-\epsilon <\ &x_n &&- x &&< \epsilon \\
		x - \epsilon < & &&x_n  &&< x + \epsilon
	\end{alignat*}
	
	Therefore, $\forall n > N$ the interval $I = (x - \epsilon, x + \epsilon)$ contains $x_n$. Because $x$ is given to be an interior point, we know that for some value $\epsilon = c$, $I = (x - c, x + c) \subset E$, so all $x_n$ are in $E$, $\forall n > N$ . 
	
\end{proof}

 
\paragraph{Exercise 4.3.9.} \textit{Show that the set $E'$ of points of accumulation of any set $E$ must be closed.}

\begin{proof}
Let $x$ be any accumulation point of $E'$. 

Assume $x \notin E'$. By the definition of accumulation points, there exists $\epsilon > 0$ such that $(x - \epsilon, x + \epsilon) \cap E$ contains at most one point, $x$, in $E$. But, since $x$ is an accumulation point of $E'$, then $\exists y \in E'$ such that $\abs{x - y} < \epsilon_1$. Similarly, $y$ is an accumulation point of $E$, so $\forall \epsilon_2$, $\exists z \in E$ such that $\abs{y - z} < \epsilon_2$. If we take the interval $\abs{x - z}$, then:
\begin{align*}
	\abs{x - z} &= \abs{x + (y - y) - z} \\
				&= \abs{(x - y) + (y - z)} \\
				&\leq \abs{x - y} + \abs{y - z} \tag{by triangle inequality} \\
				&< \epsilon_1 + \epsilon_2
\end{align*}
We can choose $\epsilon_1, \epsilon_2 < \frac{\epsilon}{2}$, which means $\abs{x - z} < \epsilon$, and every interval around $x$ contains at least two points in $E$, so $x$ is an accumulation of $E$ and $x \in E'$, which contradicts our initial assumption. Therefore, every accumulation point of $E'$ is in $E'$.

\end{proof}

\newpage
\def \AB {A \setminus B} 
\def \BA {B \setminus A} 
\def \poa {point of accumulation}
\paragraph{Exercise 4.3.13.} \textit{If $A$ is open and $B$ is closed, what can you say about the sets $\AB$ and $\BA$?} \newline

$\AB$ is open.
\begin{proof}
	to-do...
\end{proof}

$\BA$ is closed.
\begin{proof}
By the definition of an open set, every point in A must be an interior point, so for all $a \in A$: $$\exists c > 0 \text{ such that } (a - c, a + c) \subset A$$ Since $(a - c, a + c)$ is a subset of $A$, then $(a - c, a + c) \cap (\BA) = \emptyset$. The point $a$ then cannot be a {\poa} in $\BA$, and since this is true for every point in $A$, we know that none of the points in $A$ are point of accumulation of $\BA$. Clearly, we cannot create new points of accumulation by removing points from a set, therefore all points of accumulation of $\BA$ are in $\BA$, so $\BA$ is closed.
\end{proof}

\def \setxn {\{\xn\}}
\def \xnk {x_{n_k}}
\def \setxnk {\{\xnk\}}
\paragraph{Exercise 4.3.18.} \textit{Let $\{x_n\}$ be a sequence of real numbers.
Let $E$ denote the set of all numbers $z$ that have the property that there exists a subsequence $\{x_{n_k}\}$ convergent to $z$. Show that $E$ is closed.} 

\begin{proof}
Let $S$ be the set of all elements in $\setxn$. For every element $z \in E$ there exists a subsequence, $\setxnk$, such that $\setxnk$ converges to $z$. By definition of convergence, for all $\epsilon > 0$ there exists some value $n_0$ such that for all $n > n_0$, $\abs*{\xnk - z} < \epsilon$, so there are infinite points in $S$ contained in the interval $(z - \epsilon, z + \epsilon)$. Therefore $z$ must be a point of accumulation point of $S$, and $E$ is the set of all accumulation points of $S$, which in Exercise 4.3.9. we have shown is closed.
\end{proof}

\end{document}
