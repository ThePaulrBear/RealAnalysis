\documentclass[11pt]{article}

\usepackage{amsmath,amsthm}
\usepackage{amsfonts}
\usepackage{amssymb}
\usepackage{enumitem}
\usepackage{mathtools}

\DeclarePairedDelimiter{\abs}{\lvert}{\rvert}
\DeclarePairedDelimiter{\norm}{\lVert}{\rVert}

\DeclarePairedDelimiter{\bracks}{\lbrack}{\rbrack}
\DeclarePairedDelimiter{\braces}{\lbrace}{\rbrace}

% Highly-Reusable
\def \R {\mathbb{R}}
\def \epsilon {\varepsilon}
\def \halfepsilon{\frac{\epsilon}{2}}

\theoremstyle{definition}
\newtheorem{definition}{Definition}[section]

\title{Real Analysis Definitions}
\author{CS 117 Real Analysis}


\def \ab {[a, b]}
\def \halfdelta {\frac{\delta}{2}}
\def \xneighborhood {[x - \halfdelta, x + \halfdelta]}
\def \xinterval {[\ximinusone, x_i]}
\def \ximinusone {x_{i-1}}
\def \ci {c_i}

\begin{document}
\maketitle

\section{Properties of the Real Numbers}
\begin{definition}[Upper Bounds]
	Definition 1.1: (Upper Bounds) Let E be a set of real numbers. A number M is said to be an upper bound for E if x ≤ M for all x ∈ E.
\end{definition}


\begin{definition} [Lower Bounds]
	Definition 1.2: (Lower Bounds) Let E be a set of real numbers. A number m is said to be a lower
bound for E if m ≤ x for all x ∈ E
\end{definition}


\begin{definition} [Maximum]
	Definition 1.3: (Maximum) Let E be a set of real numbers. If there is a number M that belongs to E and is larger than every other member of E, then M is called the maximum of the set E and we write M = max E.
\end{definition}


\begin{definition} [Minimum]
	Definition 1.4: (Minimum) Let E be a set of real numbers. If there is a number m that belongs to E and is smaller than every other member of E, then m is called the minimum of the set E and we write m = min E.
\end{definition}


\setcounter{definition}{8}
\begin{definition} [Least Upper Bound/Supremum]
	Definition 1.9: (Least Upper Bound/Supremum) Let E be a set of real numbers that is bounded
	above and nonempty. If M is the least of all the upper bounds, then M is said to be the least upper bound of E or the supremum of E and we write M = sup E.
\end{definition}


\begin{definition} [Greatest Lower Bound/Infimum]
	Definition 1.10: (Greatest Lower Bound/Infimum) Let E be a set of real numbers that is bounded below and nonempty. If m is the greatest of all the lower bounds of E, then m is said to be the greatest lower bound of E or the infimum of E and we write M = inf E.
\end{definition}

\setcounter{definition}{13}
\begin{definition} [Dense Sets]
	Definition 1.14: (Dense Sets) A set E of real numbers is said to be dense (or dense in R) if every interval (a, b) contains a point of E.
\end{definition}

\setcounter{definition}{15}
\begin{definition} [Absolute Value]
	Definition 1.16: (Absolute Value) For any real number x write
	|x| = x if x ≥ 0
	and
	|x| = −x if x < 0 .
\end{definition}

\setcounter{definition}{17}
\begin{definition} [Distance]
	Definition 1.18: (Distance) The distance between two real numbers x and y is d(x, y) = |x − y|.
\end{definition}

\section{Sequences}

\begin{definition} 
	Definition 2.1: By a sequence of real numbers we mean a function f : IN → R.
\end{definition}

\setcounter{definition}{4}
\begin{definition} [Countable]
	Definition 2.5: (Countable) A nonempty set S of real numbers is said to be countable if there is a sequence of real numbers whose range is the set S
\end{definition}


\begin{definition} [Limit of a Sequence]
	Definition 2.6: (Limit of a Sequence) Let {sn} be a sequence of real numbers. We say that {sn}
	converges to a number L and write
	limn→∞
	sn = L
	or
	sn → L as n → ∞
	provided that for every number ε > 0 there is an integer N so that
	|sn − L| < ε
	whenever n ≥ N.
\end{definition}

\setcounter{definition}{8}
\begin{definition} [Divergence to $\infty$]
	Definition 2.9: (Divergence to ∞) Let \{sn\} be a sequence of real numbers. We say that \{sn\} diverges
	to ∞ and write
	limn→∞
	sn = ∞
	or
	sn → ∞ as n → ∞
	provided that for every number M there is an integer N so that
	sn ≥ M
	whenever n ≥ N.
\end{definition}

\setcounter{definition}{23}
\begin{definition} [Increasing]
	Definition 2.24: (Increasing) We say that a sequence {sn} is increasing if s1 < s2 < s3 < · · · < sn < sn+1 < . . . .	
\end{definition}


\begin{definition} [Decreasing]
	Definition 2.25: (Decreasing) We say that a sequence {sn} is decreasing if	s1 > s2 > s3 > · · · > sn > sn+1 > . . . .
\end{definition}


\begin{definition} [Nondecreasing]
	Definition 2.26: (Nondecreasing) We say that a sequence {sn} is nondecreasing if s1 ≤ s2 ≤ s3 ≤ · · · ≤ sn ≤ sn+1 ≤ . . . .
\end{definition}


\begin{definition} [Nonincreasing]
	Definition 2.27: (Nonincreasing) We say that a sequence {sn} is nonincreasing if s1 ≥ s2 ≥ s3 ≥ · · · ≥ sn ≥ sn+1 ≥ . . . .
\end{definition}

\setcounter{definition}{36}
\begin{definition} [Subsequences]
	Definition 2.37: (Subsequences) Let
	s1, s2, s3, s4, . . .
	be any sequence. Then by a subsequence of this sequence we mean any sequence
	sn1
	, sn2
	, sn3
	, sn4
	, . . .
	where
	n1 < n2 < n3 < . . .
	is an increasing sequence of natural numbers.
\end{definition}

\setcounter{definition}{43}
\begin{definition} [Lim Sup]
	Definition 2.44: (Lim Sup) A limit superior of a sequence {xn}, denoted as
	lim sup
	n→∞
	xn,
	is defined to be the infimum of all numbers β with the following property:
	There is an integer N so that xn < β for all n ≥ N.
\end{definition}


\begin{definition} [Lim Inf]
	Definition 2.45: (Lim Inf) A limit inferior of a sequence {xn}, denoted as
	lim inf
	n→∞
	xn,
	is defined to be the supremum of all numbers α with the following property:
	There is an integer N so that α < xn for all n ≥ N
\end{definition}

\section{Infinite Sums}

\setcounter{definition}{1}
\begin{definition}
	Definition 3.2: Let I be an infinite set and a a function a : I → R. Then we write
	X
	i∈I
	ai = c
	and say that the sum converges if for every ε > 0 there is a finite set I0 ⊂ I so that, for every finite set J,
	I0 ⊂ J ⊂ I,
	
	X
	i∈J
	ai − c
	
	< ε.
	A sum that does not converge is said to diverge.
\end{definition}

\setcounter{definition}{5}
\begin{definition}
	Definition 3.6: Let {ak} be a sequence of real numbers. Then we write
	X∞
	k=1
	ak = c
	and say that the series converges if the sequence
	sn =
	Xn
	k=1
	ak
	(called the sequence of partial sums of the series) converges to c. If the series does not converge it is said
	to be divergent.
\end{definition}

\setcounter{definition}{12}
\begin{definition} [Cauchy Criterion for Convergence]
	Definition 3.13: The series
	X∞
	k=1
	ak
	is said to satisfy the Cauchy criterion for convergence provided that for every ε > 0 there is an integer N
	so that all of the finite sums
	
	Xm
	k=n
	ak
	
	< ε
	for any N ≤ n < m < ∞.
\end{definition}

\setcounter{definition}{14}
\begin{definition} [Absolutely Convergent]
	Definition 3.15: A series P∞
	k=1 ak is said to be absolutely convergent if the related series P∞
	k=1 |ak| converges.
\end{definition}


\begin{definition} [Nonabsolutely Convergent]
	Definition 3.16: A series P∞
	k=1 ak is said to be nonabsolutely convergent if the series P∞
	k=1 ak converges
	but the series P∞
	k=1 |ak| diverges.
\end{definition}

\setcounter{definition}{50}
\begin{definition}
	Definition 3.51: The series
	X∞
	k=0
	ck
	is called the formal product of the two series
	X∞
	k=0
	ak and X∞
	k=0
	bk
	provided that
	ck =
	X
	k
	i=0
	aibk−i
	.
\end{definition}

\section{Sets of Real Numbers}

\begin{definition} [Interior Point]
	Definition 4.1: (Interior Point) Let E be a set of real numbers. Any point x that belongs to E is said
	to be an interior point of E provided that some interval
	(x − c, x + c) ⊂ E.
\end{definition}

\setcounter{definition}{2}
\begin{definition} [Isolated Point]
	Definition 4.3: (Isolated Point) Let E be a set of real numbers. Any point x that belongs to E is said
	to be an isolated point of E provided that for some interval (x − c, x + c)
	(x − c, x + c) ∩ E = {x}.
\end{definition}

\setcounter{definition}{4}
\begin{definition} [Accumulation Point]
	Definition 4.5: (Accumulation Point) Let E be a set of real numbers. Any point x (not necessarily in
	E) is said to be an accumulation point of E provided that for every c > 0 the intersection
	(x − c, x + c) ∩ E
	contains infinitely many points.
\end{definition}

\setcounter{definition}{6}
\begin{definition} [Boundary Point]
	Definition 4.7: (Boundary Point) Let E be a set of real numbers. Any point x (not necessarily in E)
	is said to be a boundary point of E provided that every interval (x − c, x + c) contains at least one point of
	E and also at least one point that does not belong to E.
\end{definition}

\setcounter{definition}{8}
\begin{definition} [Closed]
	Definition 4.9: (Closed) Let E be a set of real numbers. The set E is said to be closed provided that
	every accumulation point of E belongs to the set E.
\end{definition}

\setcounter{definition}{10}
\begin{definition} [Closure]
	Definition 4.11: (Closure) Let E be any set of real numbers and let E′ denote the set of all accumulation
	points of E. Then the set
	E = E ∪ E
	′
	is called the closure of the set E.
\end{definition}


\begin{definition} [Open]
	Definition 4.12: (Open) Let E be a set of real numbers. Then E is said to be open if every point of E
	is also an interior point of E.
\end{definition}

\setcounter{definition}{13}
\begin{definition} [Interior]
	Definition 4.14: (Interior) Let E be any set of real numbers. Then the set
	int(E)
	denotes the set of all interior points of E and is called the interior of the set E.
\end{definition}

\setcounter{definition}{26}
\begin{definition} [Cousin Cover]
	Definition 4.27: (Cousin Cover) A collection C of closed intervals satisfying the hypothesis of Cousin’s
	lemma is called a Cousin cover of [a, b].
\end{definition}


\begin{definition} [Open Cover]
	Definition 4.28: (Open Cover) Let A ⊂ R and let U be a family of open intervals. If for every x ∈ A
	there exists at least one interval U ∈ U such that x ∈ U, then U is called an open cover of A.
\end{definition}


\begin{definition} [Heine-Borel Property]
	Definition 4.29: (Heine-Borel Property) A set A ⊂ R is said to have the Heine-Borel property if every
	open cover of A can be reduced to a finite subcover. That is, if U is an open cover of A, then there exists a
	finite subset of U, {U1, U2, . . . , Un} such that
	A ⊂ U1 ∪ U2 ∪ · · · ∪ Un.
\end{definition}

\setcounter{definition}{33}
\begin{definition} [Compact]
	Definition 4.34: A set of real numbers E is said to be compact if it has any of the following equivalent
	properties:
	1. E is closed and bounded.
	2. E has the Bolzano-Weierstrass property.
	3. E has the Heine-Borel property
\end{definition}

\section{Continuous Functions}

\begin{definition} [Limit - ε-δ Definition]
	Definition 5.1: (Limit) Let f : E → R be a function with domain E and suppose that x0 is a point of
	accumulation of E. Then we write
	limx→x0
	f(x) = L
	if for every ε > 0 there is a δ > 0 so that
	|f(x) − L| < ε
	whenever x is a point of E differing from x0 and satisfying |x − x0| < δ.
\end{definition}

\setcounter{definition}{3}
\begin{definition} [Limit - Sequential Definition]
	Definition 5.4: (Limit) Let f : E → R be a function with domain E and suppose that x0 is a point of
	accumulation of E. Then we write
	limx→x0
	f(x) = L
	if for every sequence {en} of points of E with en 6= x0 and en → x0 as n → ∞,
	limn→∞
	f(en) = L.
\end{definition}

\setcounter{definition}{5}
\begin{definition} [Limit - Mapping Definition]
	Definition 5.6: (Limit) Let f : E → R be a function with domain E and suppose that x0 is a point of
	accumulation of E. Then we write
	limx→x0
	f(x) = L
	if for every open set V containing the point L there is an open set U containing the point x0 and every
	point x 6= x0 of U that is in the domain of f is mapped into a point in V ; that is,
	f : E ∩ U \ {x0} → V.
\end{definition}


\begin{definition} [Right-Hand Limit]
	Definition 5.7: (Right-Hand Limit) Let f : E → R be a function with domain E and suppose that x0
	is a point of accumulation of E ∩ (x0, ∞). Then we write
	lim x→x0+
	f(x) = L
	if for every ε > 0 there is a δ > 0 so that
	|f(x) − L| < ε
	whenever x0 < x < x0 + δ and x ∈ E.
\end{definition}


\begin{definition} [Right-Hand Limit]
	Definition 5.8: (Right-Hand Limit) Let f : E → R be a function with domain E and suppose that x0
	is a point of accumulation of E ∩ (x0, ∞). Then we write
	lim x→x0+
	f(x) = L
	if for every decreasing sequence {en} of points of E with en > x0 and en → x0 as n → ∞,
	limn→∞
	f(en) = L.
\end{definition}


\begin{definition} [Infinite Limit]
	Definition 5.9: (Infinite Limit) Let f : E → R be a function with domain E and suppose that x0 is a
	point of accumulation of E ∩ (x0, ∞). Then we write
	lim x→x0+
	f(x) = ∞
	if for every M > 0 there is a δ > 0 so that f(x) ≥ M whenever
	x0 < x < x0 + δ and x ∈ E.
\end{definition}

\setcounter{definition}{25}
\begin{definition} [Lim Sup and Lim Inf]
	Definition 5.26: (Lim Sup and Lim Inf) Let f : E → R be a function with domain E and suppose
	that x0 is a point of accumulation of E. Then we write
	lim sup
	x→x0
	f(x) = inf
	δ>0
	sup{f(x) : x ∈ (x0 − δ, x0 + δ) ∩ E, x 6= x0}
	and
	lim inf
	x→x0
	f(x) = sup
	δ>0
	inf{f(x) : x ∈ (x0 − δ, x0 + δ) ∩ E, x 6= x0}
\end{definition}


\begin{definition} [Intermediate Value Property]
	Definition 5.27: (Intermediate Value Property) Let f be defined on an interval I. Suppose that for
	each a, b ∈ I with f(a) 6= f(b), and for each d between f(a) and f(b), there exists c between a and b for
	which f(c) = d. We then say that f has the intermediate value property (IVP) on I
\end{definition}

\setcounter{definition}{28}
\begin{definition} [Continuous]
	Definition 5.29: (Continuous) Let f be defined in a neighborhood of x0. The function f is continuous
	at x0 provided limx→x0
	f(x) = f(x0).
\end{definition}

\setcounter{definition}{31}
\begin{definition} [$\epsilon - \delta$ Version]
	Definition 5.32: (ε-δ Version) Let f be defined on a set A and let x0 be any point of A. The function
	f is continuous at x0 provided for every ε > 0 there is a δ > 0 so that
	|f(x) − f(x0)| < ε
	for every x ∈ A for which |x − x0| < δ.
\end{definition}


\begin{definition} [Limit Version]
	Definition 5.33: (Limit Version) Let f be defined on a set A and let x0 be any point of A. The function
	f is continuous at x0 provided either that x0 is isolated in A or else that x0 is a point of accumulation of
	that set and
	limx→x0
	f(x) = f(x0).
\end{definition}


\begin{definition} [Neighborhood Version]
	Definition 5.34: (Neighborhood Version) Let f be defined on a set A and let x0 be any point of A.
	The function f is continuous at x0 provided that for every open set V containing f(x0) there is an open set
	U containing x0 so that f(U ∩ A) ⊂ V .
\end{definition}


\begin{definition} [Sequential Version]
	Definition 5.35: (Sequential Version) Let f be defined on a set A and let x0 be any point of A. The
	function f is continuous at x0 provided that for every sequence of points {xn} belonging to A and converging
	to x0, it follows that f(xn) → f(x0).
\end{definition}


\begin{definition}
	Definition 5.36: Let f : A → R. Then f is continuous (or continuous on A) if f is continuous at each
	point of A.
\end{definition}

\setcounter{definition}{45}
\begin{definition} [Uniformly Continuous]
	Definition 5.46: (Uniformly Continuous) Let f be defined on a set A ⊂ R. We say that f is uniformly
	continuous (on A) if for every ε > 0 there exists δ > 0 such that if x, y ∈ A and |x − y| < δ, then
	|f(x) − f(y)| < ε.
\end{definition}

\setcounter{definition}{55}
\begin{definition} [Nondecreasing]
	Definition 5.56: (Nondecreasing) Let f be real valued on an interval I. If f(x1) ≤ f(x2) whenever x1
	and x2 are points in I with x1 < x2, we say f is nondecreasing on I.
\end{definition}


\begin{definition} [Increasing]
	Definition 5.57: (Increasing) Let f be real valued on an interval I. If the strict inequality f(x1) < f(x2)
	holds whenever x1 and x2 are points in I with x1 < x2, we say f is increasing on I.
\end{definition}


\begin{definition} [Nonincreasing]
	Definition 5.58: (Nonincreasing) Let f be real valued on an interval I. If f(x1) ≥ f(x2) whenever x1
	and x2 are points in I with x1 < x2, we say f is nonincreasing on I.
\end{definition}


\begin{definition} [Decreasing]
	Definition 5.59: (Decreasing) Let f be real valued on an interval I. If the strict inequality f(x1) > f(x2)
	holds whenever x1 and x2 are points in I with x1 < x2, we say f is decreasing on I.
\end{definition}


\end{document}