\documentclass[11pt]{article}

\usepackage{amsmath,amsthm}
\usepackage{amsfonts}
\usepackage{amssymb}
\usepackage{enumitem}
\usepackage{mathtools}
\usepackage{listofitems}	

\DeclarePairedDelimiter{\abs}{\lvert}{\rvert}
\DeclarePairedDelimiter{\norm}{\lVert}{\rVert}

\DeclarePairedDelimiter{\bracks}{\lbrack}{\rbrack}
\DeclarePairedDelimiter{\braces}{\lbrace}{\rbrace}

\newcommand\textlist[3][\cr]{%
	\readlist\indices{#3}%
	\foreachitem\x\in\indices{%
		\ifnum\xcnt=1\else, \fi$#2_{\x}$%
	}%
	\ifx\cr#1\relax, \ldots\else%
	\if\relax#1\relax\else, \ldots, $#2_{#1}$\fi%
	\fi%
}

% Highly-Reusable
\def \R {\mathbb{R}}
\def \N {\mathbb{N}}
\def \epsilon {\varepsilon}
\newcommand{\set}[1]{\{#1\}}
\newcommand{\limittoinfy}[1]{\lim_{#1 \to \infty}}
\def \limittoinfyn {\lim_{n \to \infty}}
\def \limsup {\lim_{n \to \infty} \sup}
\def \liminf {\lim_{n \to \infty} \inf}

\def \limitToXZero {\lim_{x \to x_0}}
\def \limsupX0 {\underset{x \to x_0}{\lim \sup}}
\def \liminfX0 {\underset{x \to x_0}{\lim \inf}}
\def \limToX0FromRight {\lim_{x \to x_0+}}

\newcommand{\sumofto}[2]{\sum_{#1=1}^{#2}}
\newcommand{\sumnto}[1]{\sumofto{n}{#1}}
\newcommand{\sumoftoinfty}[1]{\sumofto{#1}{\infty}}
\def \sumntoinfy {\sumnto{\infty}}

\def \sn {s_n}
\def \snplusone {s_{n+1}}
\def \seqsn{ \set{\sn} }
\def \tn {t_n}
\def \xn {x_n}
\def \an {a_n}
\def \en {e_n}
\def \bn {b_n}
\def \ak {a_k}

\def \scriptC {\mathcal{C}}
\def \scriptU {\mathcal{U}}

\theoremstyle{definition}
\newtheorem{definition}{Definition}[section]

\title{Real Analysis Definitions}
\author{CS 117 Real Analysis}


\def \ab {[a, b]}
\def \halfdelta {\frac{\delta}{2}}
\def \xneighborhood {[x - \halfdelta, x + \halfdelta]}
\def \xinterval {[\ximinusone, x_i]}
\def \ximinusone {x_{i-1}}
\def \ci {c_i}

\begin{document}
\maketitle

\section{Properties of the Real Numbers}
\begin{definition}[Upper Bounds]
	Let $E$ be a set of real numbers. A number $ M $ is said to be an \textit{upper bound} for $ E $ if $ x \leq M $ for all $ x \in E $.
\end{definition}


\begin{definition} [Lower Bounds]
	Let $ E $ be a set of real numbers. A number $ m $ is said to be a \textit{lower bound} for $ E $ if $ m \leq x $ for all $ x \in E $
\end{definition}


\begin{definition} [Maximum]
	Let $ E $ be a set of real numbers. If there is a number $ M $ that belongs to $ E $ and is larger than every other member of $ E $, then $ M $ is called the \textit{maximum} of the set $ E $ and we write $ M = \max E $.
\end{definition}


\begin{definition} [Minimum]
	Let $ E $ be a set of real numbers. If there is a number $ m $ that belongs to $ E $ and is smaller than every other member of $ E $, then $ m $ is called the \textit{minimum} of the set $ E $ and we write $ m = \min E $.
\end{definition}


\setcounter{definition}{8}
\begin{definition} [Least Upper Bound/Supremum]
	Let $ E $ be a set of real numbers that is bounded above and nonempty. If $ M $ is the least of all the upper bounds, then $ M $ is said to be the \textit{least upper bound} of $ E $ or the \textit{supremum} of $ E $ and we write $ M = \sup E $.
\end{definition}


\begin{definition} [Greatest Lower Bound/Infimum]
	Let $ E $ be a set of real numbers that is bounded below and nonempty. If $ m $ is the greatest of all the lower bounds of $ E $, then $ m $ is said to be the \textit{greatest lower bound} of $ E $ or the \textit{infimum} of $ E $ and we write $ M = \inf E $.
\end{definition}

\setcounter{definition}{13}
\begin{definition} [Dense Sets]
	A set $ E $ of real numbers is said to be \textit{dense} (or \textit{dense in $\R$}) if every interval $ (a, b) $ contains a point of $ E $.
\end{definition}

\setcounter{definition}{15}
\begin{definition} [Absolute Value]
	For any real number $ x $ write	
	\[ |x| =
	\begin{cases}
		x & \quad \text{if } x \geq 0 \\
		-x & \quad \text{if } x < 0
	\end{cases}
	\]
\end{definition}

\setcounter{definition}{17}
\begin{definition} [Distance]
	The \textit{distance} between two real numbers $ x $ and $ y $ is $d(x, y) = |x - y|$.
\end{definition}

\section{Sequences}

\begin{definition} [Sequence]
	By a \textit{sequence} of real numbers we mean a function $ f : \N \rightarrow \R $.
\end{definition}

\setcounter{definition}{4}
\begin{definition} [Countable]
	A nonempty set $ S $ of real numbers is said to be \textit{countable} if there is a sequence of real numbers whose range is the set $ S $.
\end{definition}


\begin{definition} [Limit of a Sequence]
	Let $ \seqsn $ be a sequence of real numbers. We say that $ \seqsn $	\textit{converges} to a number $ L $ and write $$ \limittoinfyn \sn = L $$
	or
	$$\sn \rightarrow  L \text{ as } n \rightarrow \infty$$
	provided that for every number $ \epsilon > 0 $ there is an integer $ N $ so that $$ |\sn - L| < \epsilon $$ whenever $ n \geq N $.
\end{definition}

\setcounter{definition}{8}
\begin{definition} [Divergence to $\infty$]
	Let $ \seqsn $ be a sequence of real numbers. We say that $ \seqsn $ \textit{diverges} to $ \infty $ and write
	$$ \limittoinfyn \sn = \infty $$ or	$$ \sn \rightarrow \infty \text{ as } n \rightarrow \infty $$	provided that for every number $ M $ there is an integer $ N $ so that $ \sn \geq M $
	whenever $ n \geq N $.
\end{definition}

\setcounter{definition}{23}
\begin{definition} [Increasing]
	We say that a sequence $ \seqsn $ is \textit{increasing} if $$ s_1 < s_2 < s_3 < \dots < \sn < s_{n+1} < \dots $$
\end{definition}


\begin{definition} [Decreasing]
	We say that a sequence $ \seqsn $ is \textit{decreasing} if	$$ s_1 > s_2 > s_3 > \dots > s_n > \snplusone > \dots $$
\end{definition}


\begin{definition} [Nondecreasing]
	We say that a sequence $ \seqsn $ is \textit{nondecreasing} if $$ s_1 \leq s_2 \leq s_3 \leq \dots \leq \sn \leq \snplusone \leq \dots$$
\end{definition}


\begin{definition} [Nonincreasing]
	We say that a sequence $ \seqsn $ is \textit{nonincreasing} if $$ s_1 \geq s_2 \geq s_3 \geq \dots \geq \sn \geq \snplusone \geq \dots $$
\end{definition}

\setcounter{definition}{36}
\begin{definition} [Subsequences]
	Let \textlist{s}{1, 2, 3} be any sequence. Then by a \textit{subsequence} of this sequence we mean any sequence \textlist{s}{n_1, n_2, n_3} where $n_1 < n_2 < n_3 < \dots$ is an increasing sequence of natural numbers.
\end{definition}

\setcounter{definition}{43}
\begin{definition} [Limit Superior]
	A \textit{limit superior} of a sequence $\set{\xn}$, denoted as	$\limsup \xn$, is defined to be the infimum of all numbers $ \beta $ with the following property:
	
	There is an integer $ N $ so that $ \xn < \beta $ for all $ n \geq N $.
\end{definition}

\begin{definition} [Limit Inferior]
	A \textit{limit inferior} of a sequence $\set{\xn}$, denoted as	$\liminf \xn$, is defined to be the supremum of all numbers $ \alpha $ with the following property:
	
	There is an integer $ N $ so that $ \alpha  < \xn$ for all $ n \geq N $
\end{definition}

\section{Infinite Sums}

\setcounter{definition}{1}
\begin{definition}
	Let $ I $ be an infinite set and $ a $ a function $ a : I \rightarrow \R $. Then we write
	$$\sum_{i \in I} a_i = c$$ 
	and say that the sum \textit{converges} if for every $ \epsilon > 0 $ there is a finite set $ I_0 \subset I $ so that, for every finite set $ J $, $ I_0 \subset J \subset I $,
	
	$$\abs*{\sum_{i \in I} a_i - c} < \epsilon$$ 

	A sum that does not converge is said to \textit{diverge}.
\end{definition}

\setcounter{definition}{5}
\begin{definition}
	Let $ \set{\ak} $ be a sequence of real numbers. Then we write 
		$$ \sumofto{k}{\infty} a_k = c $$
	and say that the series \textit{converges} if the sequence 
		$$\sn = \sumofto{k}{n} \ak$$
	(called the sequence of partial sums of the series) converges to $ c $. If the series does not converge it is said to be $ divergent $.
\end{definition}


\setcounter{definition}{13}
\begin{definition} [Cauchy Criterion for Convergence]
	The series $$ \sumoftoinfty{k} \ak $$ is said to satisfy the \textit{Cauchy criterion for convergence} provided that for every $ \epsilon > 0 $ there is an integer $ N $ so that all of the finite sums
	$$\abs*{\sum_{k=n}^{m} \ak} < \epsilon$$ for any $ N \leq n < m < \infty $.
\end{definition}


\begin{definition} [Absolutely Convergent]
	A series $\sumoftoinfty{k} \ak$ is said to be \textit{absolutely convergent} if the related series $\sumoftoinfty{k} \abs{\ak}$ converges.
\end{definition}


\begin{definition} [Nonabsolutely Convergent]
	A series $\sumoftoinfty{k} \ak$ is said to be \textit{nonabsolutely convergent} if the series $\sumoftoinfty{k} \ak$ converges but the series $\sumoftoinfty{k} \abs{\ak}$ diverges.
\end{definition}

\section{Sets of Real Numbers}

\begin{definition} [Interior Point]
	Let $ E $ be a set of real numbers. Any point $ x $ that belongs to $ E $ is said to be an \textit{interior point} of $ E $ provided that some interval $ (x - c, x + c) \subset E $.
\end{definition}

\setcounter{definition}{2}
\begin{definition} [Isolated Point]
	Let $ E $ be a set of real numbers. Any point $ x $ that belongs to $ E $ is said to be an \textit{isolated point} of $ E $ provided that for some interval $(x - c, x + c)$
		$$(x - c, x + c) \cap E = \set{x}.$$
\end{definition}

\setcounter{definition}{4}
\begin{definition} [Accumulation Point]
	Let $ E $ be a set of real numbers. Any point $ x $ (not necessarily in	$ E $) is said to be an \textit{accumulation point} of $ E $ provided that for every $ c > 0 $ the intersection
		$$ (x - c, x + c) \cap E $$
	contains infinitely many points.
\end{definition}

\setcounter{definition}{6}
\begin{definition} [Boundary Point]
	Let $ E $ be a set of real numbers. Any point $ x $ (not necessarily in $ E $) is said to be a \textit{boundary point} of $ E $ provided that every interval $ (x - c, x + c) $ contains at least one point of $ E $ and also at least one point that does not belong to $ E $.
\end{definition}

\setcounter{definition}{8}
\begin{definition} [Closed]
	Let $ E $ be a set of real numbers. The set $ E $ is said to be \textit{closed} provided that every accumulation point of $ E $ belongs to the set $ E $.
\end{definition}

\setcounter{definition}{10}
\begin{definition} [Closure]
	Let $ E $ be any set of real numbers and let $ E' $ denote the set of all accumulation points of $ E $. Then the set
		$$ \overline{E} = E \cup E'$$
	is called the \textit{closure} of the set $ E $.
\end{definition}


\begin{definition} [Open]
	Let $ E $ be a set of real numbers. Then $ E $ is said to be \textit{open} if every point of $ E $
	is also an interior point of $ E $.
\end{definition}

\setcounter{definition}{13}
\begin{definition} [Interior]
	Let $ E $ be any set of real numbers. Then the set int(E) denotes the set of all interior points of $ E $ and is called the \textit{interior} of the set $ E $.
\end{definition}

\setcounter{definition}{26}
\begin{definition} [Cousin Cover]
%	TODO: Add Cousin's Lemma
	A collection $ \scriptC $ of closed intervals satisfying the hypothesis of Cousin’s
	lemma is called a \textit{Cousin cover} of $ [a, b] $.
\end{definition}

\begin{definition} [Open Cover]
	Let $ A \subset \R $ and let $ \scriptU $ be a family of open intervals. If for every $ x \in A $ there exists at least one interval $ U \in \scriptU $ such that $ x \in U $, then $ \scriptU $ is called an \textit{open cover} of $ A $.
\end{definition}


\begin{definition} [Heine-Borel Property]
	A set $ A \subset \R $ is said to have the \textit{Heine-Borel property} if every open cover of $ A $ can be reduced to a finite subcover. That is, if $ \scriptU $ is an open cover of $ A $, then there exists a
	finite subset of $ \scriptU $, $ {U1, U2, \dots , Un} $ such that  
	$$ A \subset U_1 \cup U_2 \cup \dots \cup U_n.$$
\end{definition}

\setcounter{definition}{33}
\begin{definition} [Compact]
	A set of real numbers $ E $ is said to be \textit{compact} if it has any of the following equivalent
	properties:

	\begin{enumerate}
		\item $ E $ is closed and bounded.
		\item $ E $ has the Bolzano-Weierstrass property.
		\item $ E $ has the Heine-Borel property.
	\end{enumerate}
	
\end{definition}


\section{Continuous Functions}

\begin{definition} [Limit --- $ \epsilon-\delta $ Definition]
	Definition 5.1: (Limit) Let $ f :  E  \rightarrow \R $ be a function with domain $ E $ and suppose that $ x_0 $ is a point of accumulation of $ E $. Then we write $$ \lim_{x \rightarrow x_0} f(x) = L$$	if for every $ \epsilon > 0 $ there is a $ \delta > 0 $ so that
		$$ |f(x) - L| < \epsilon $$
	whenever x is a point of $ E $ differing from $ x_0 $ and satisfying $ |x - x_0| < \delta $.
\end{definition}

\setcounter{definition}{3}
\begin{definition} [Limit --- Sequential Definition]
	Let $ f : E \rightarrow \R $ be a function with domain $E$ and suppose that $ x_0 $ is a point of accumulation of $ E $. Then we write $$ \limitToXZero f(x) = L $$ if for every sequence $ \set{\en} $ of points of $ E $ with $ \en \ne x_0$ and $ \en \rightarrow x_0 $ as $ n \rightarrow \infty $,
	$$ \limittoinfyn f(\en) = L$$
\end{definition}

\setcounter{definition}{5}
\begin{definition} [Limit --- Mapping Definition]
	Let $ f : E \rightarrow \R $ be a function with domain $ E $ and suppose that $ x_0 $ is a point of accumulation of $ E $. Then we write $$ \limitToXZero f(x) = L $$ if for every open set $ V $ containing the point $ L $ there is an open set $ U $ containing the point $ x_0 $ and every
	point $ x \ne x_0 $ of $ U $ that is in the domain of $ f $ is mapped into a point in $ V $; that is,
	$$ f : E \cap U \ {x_0} \rightarrow V $$
\end{definition}

\begin{definition} [Right-Hand Limit --- $ \epsilon-\delta $ Definition]
	Let $ f : E \rightarrow \R $ be a function with domain $ E $ and suppose that $ x_0 $
	is a point of accumulation of $ E \cap (x_0, \infty) $. Then we write 
		$$ \limToX0FromRight f(x) = L $$
	if for every $ \epsilon > 0 $ there is a $ \delta > 0 $ so that $ |f(x) - L| < \epsilon $ whenever $ x_0 < x < x_0 + \delta $ and $ x \in E $.
\end{definition}


\begin{definition} [Right-Hand Limit --- Sequential Definition]
	Let $ f : E \rightarrow \R $ be a function with domain $ E $ and suppose that $ x_0 $ is a point of accumulation of $ E \cap (x_0, \infty) $. Then we write $$ \limToX0FromRight f(x) = L $$ if for every decreasing sequence $\set{\en}$ of points of $ E $ with $ \en > x_0 $ and $ \en \rightarrow x_0 $ as $ n \rightarrow \infty $, 
		$$\limittoinfyn f(\en) = L $$.
\end{definition}


\begin{definition} [Infinite Limit]
	Let $ f : E \rightarrow \R $ be a function with domain $E$ and suppose that $ x_0 $ is a point of accumulation of $ E \cap (x_0, \infty) $. Then we write
		$$ \limToX0FromRight f(x) = \infty$$ 
	if for every $ M > 0 $ there is a $ \delta > 0 $ so that $ f(x) \geq M $ whenever $ x_0 < x < x_0 + \delta $ and $ x \in E $.
\end{definition}


\setcounter{definition}{25}
\begin{definition} [Limit Superior and Limit Inferior]
	Let $ f : E \rightarrow \R $ be a function with domain $E$ and suppose that $ x_0 $ is a point of accumulation of $ E $. Then we write 
		$$ \limsupX0 f(x) = \underset{\delta>0}{\inf} \sup \set{f(x) : x \in (x_0 - \delta, x_0 + \delta) \cap E, x \ne x_0}$$
	and
		$$ \liminfX0 f(x) = \underset{\delta>0}{\sup} \inf \set{f(x) : x \in (x_0 - \delta, x_0 + \delta) \cap E, x \ne x_0}$$
\end{definition}


\begin{definition} [Intermediate Value Property]
	Let $ f $ be defined on an interval $ I $. Suppose that for	each $ a, b \in I $ with $ f(a) \ne f(b) $, and for each $ d $ between $ f(a) $ and $ f(b) $, there exists $ c $ between $ a $ and $ b $ for which $ f(c) = d $. We then say that $ f $ has the \textit{intermediate value property} (IVP) on $ I $.
\end{definition}

\setcounter{definition}{28}
\begin{definition} [Continuous on neighborhood at $ x_0 $  --- Limit Version]
	Let f be defined in a neighborhood of $ x_0 $. The function $ f $ is \textit{continuous
	at $ x_0 $} provided $$ \limitToXZero f(x) = f(x_0) $$
\end{definition}

\setcounter{definition}{31}
\begin{definition} [Continuous function on a set at $ x_0 $ --- $\epsilon - \delta$ Version]
	Let $ f $ be defined on a set $ A $ and let $ x_0 $ be any point of $ A $. The function
	$ f $ is \textit{continuous at $ x_0 $} provided for every $ \epsilon > 0 $ there is a $ \delta > 0 $ so that
	$ |f(x) - f(x_0)| < \epsilon $
	for every $ x \in A $ for which $ |x - x_0| < \delta $.
\end{definition}


\begin{definition} [Continuous function on a set at $ x_0 $ --- Limit Version]
	Let $ f $ be defined on a set $ A $ and let $ x_0 $ be any point of $ A $. The function
	$ f $ is \textit{continuous at $ x_0 $} provided either that $ x_0 $ is isolated in $ A $ or else that $ x_0 $ is a point of accumulation of that set and $$ \limitToXZero f(x) = f(x_0) $$
\end{definition}


\begin{definition} [Continuous function on a set at $ x_0 $ --- Neighborhood Version]
	Let $ f $ be defined on a set $ A $ and let $ x_0 $ be any point of $ A $. The function $ f $ is \textit{continuous at $ x_0 $} provided that for every open set $ V $ containing $ f(x_0) $ there is an open set $ U $ containing $ x_0 $ so that $ f(U \cap A) \subset V $.
\end{definition}


\begin{definition} [Sequential Version]
	Let $ f $ be defined on a set $ A $ and let $ x_0 $ be any point of $ A $. The function $ f $ is \textit{continuous at $ x_0 $} provided that for every sequence of points $ \set{\xn} $ belonging to $ A $ and converging to $ x_0 $, it follows that $ f(\xn) \rightarrow f(x_0) $.
\end{definition}


\begin{definition} [Continuous Function]
	Let $ f : A \rightarrow \R $. Then f is \textit{continuous} (or \textit{continuous on $ A $}) if f is continuous at each point of $ A $.
\end{definition}

\setcounter{definition}{45}
\begin{definition} [Uniformly Continuous]
	Let $ f $ be defined on a set $ A \subset \R $. We say that $ f $ is \textit{uniformly
	continuous} (on A) if for every $ \epsilon > 0 $ there exists $ \delta > 0 $ such that if $ x, y \in A $ and $ |x - y| < \delta $, then $ |f(x) - f(y)| < \epsilon $.
\end{definition}

\setcounter{definition}{55}
\begin{definition} [Nondecreasing]
	Let $ f $ be real valued on an interval $ I $. If $ f(x_1) \leq f(x_2) $ whenever $ x_1 $
	and $ x_2 $ are points in $ I $ with $ x_1 < x_2 $, we say $ f $ is \textit{nondecreasing} on $ I $.
\end{definition}


\begin{definition} [Increasing]
	Let $ f $ be real valued on an interval $ I $. If the strict inequality $ f(x_1) < f(x_2) $ holds whenever $ x_1 $ and $ x_2 $ are points in $ I $ with $ x_1 < x_2 $, we say $ f $ is \textit{increasing} on $ I $.
\end{definition}


\begin{definition} [Nonincreasing]
	Let $ f $ be real valued on an interval $ I $. If $ f(x_1) \geq f(x_2) $ whenever $ x_1 $ and $ x_2 $ are points in $ I $ with $ x_1 < x_2 $, we say $ f $ is \textit{nonincreasing} on $ I $.
\end{definition}


\begin{definition} [Decreasing]
	Let $ f $ be real valued on an interval $ I $. If the strict inequality $ f(x_1) > f(x_2) $ holds whenever $ x_1 $ and $ x_2 $ are points in $ I $ with $ x_1 < x_2 $, we say $ f $ is \textit{decreasing} on $ I $.
\end{definition}


\setcounter{section}{6}
\section{Differentiation}

\begin{definition} [Derivative]
	Let $ f $ be defined on an interval $ I $ and let $ x_0 \in I. $ The \textit{derivative} of $ f $ at $ x_0 $, denoted by $ f'(x_0) $, is defined as
	$$ f'(x_0) = \limitToXZero \frac{f(x) - f(x_0)}{x - x_0} $$ provided either that this limit exists or is infinite. If $f'(x_0)$ is finite we say that $ f $ is $ differentiable $ at $ x_0 $. If $ f $ is differentiable at every point of a set $ E \subset I, $ we say that $ f $ is \textit{differentiable} on $ E $. When $ E $ is all of $ I $, we simply say that $ f $ is a \textit{differentiable function}.
\end{definition}

\begin{definition} [Derivative]
	Let $ f $ be defined on an interval $ I $ and let $ x_0 \in I$. The \textit{right-hand derivative} of $ f $ at $ x_0 $,	denoted by $ f'_+(x_0) $ is the limit
		$$ f'_+(x_0) = \lim_{x \to x_0+} \frac{f(x) - f(x_0)}{x - x_0},$$
	provided that one-sided limit exists or is infinite. 
	
	Similarly, the \textit{left-hand derivative} of $ f $ at $ x_0 $, $ f'_-(x_0)$, is
	the limit 
		$$f'_-(x_0) = \lim_{x \to x_0-} \frac{f(x) - f(x_0)}{x - x_0}.$$

\end{definition}



\end{document}