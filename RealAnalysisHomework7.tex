\documentclass[11pt]{article}

\usepackage{amsmath,amsthm}
\usepackage{amsfonts}
\usepackage{amssymb}
\usepackage{enumitem}
\usepackage{mathtools}

\DeclarePairedDelimiter{\abs}{\lvert}{\rvert}
\DeclarePairedDelimiter{\norm}{\lVert}{\rVert}

\DeclarePairedDelimiter{\bracks}{\lbrack}{\rbrack}
\DeclarePairedDelimiter{\braces}{\lbrace}{\rbrace}

% Lists
\newcommand\textlist[3][\cr]{%
	\readlist\indices{#3}%
	\foreachitem\x\in\indices{%
		\ifnum\xcnt=1\else, \fi$#2_{\x}$%
	}%
	\ifx\cr#1\relax, \ldots\else%
	\if\relax#1\relax\else, \ldots, $#2_{#1}$\fi%
	\fi%
}

% Highly-Reusable
\def \R {\mathbb{R}}
\def \N {\mathbb{N}}
\def \epsilon {\varepsilon}
\newcommand{\set}[1]{\{#1\}}
\newcommand{\limittoinfy}[1]{\lim_{#1 \to \infty}}
\def \limittoinfyn {\lim_{n \to \infty}}
\def \limsup {\lim_{n \to \infty} \sup}
\def \liminf {\lim_{n \to \infty} \inf}

\def \limitToXZero {\lim_{x \to x_0}}
\def \limsupX0 {\underset{x \to x_0}{\lim \sup}}
\def \liminfX0 {\underset{x \to x_0}{\lim \inf}}
\def \limToX0FromRight {\lim_{x \to x_0+}}

\newcommand{\sumofto}[2]{\sum_{#1=1}^{#2}}
\newcommand{\sumnto}[1]{\sumofto{n}{#1}}
\newcommand{\sumoftoinfty}[1]{\sumofto{#1}{\infty}}
\def \sumntoinfy {\sumnto{\infty}}

\def \sn {s_n}
\def \snplusone {s_{n+1}}
\def \seqsn{ \set{\sn} }
\def \tn {t_n}
\def \xn {x_n}
\def \an {a_n}
\def \en {e_n}
\def \bn {b_n}
\def \ak {a_k}

\def \scriptC {\mathcal{C}}
\def \scriptU {\mathcal{U}}

\newtheorem{theorem}{Theorem}[]

\title{Assignment 7}
\author{Paul Wintz\\117 Real Analysis}
\date{March 5, 2018}

\begin{document}
\maketitle


\def \L {\sqrt{x_0}}
\def \M {\sqrt{x_0 + 1} + \L}
\section*{5.1 \textsl{}Introduction to Limits}
\paragraph{Exercise 5.1.8.} Prove the validity of the limit $\lim_{x \rightarrow x_0} \sqrt{x} = \L$.

\begin{proof}

If the limit exists, then for any $\epsilon_0 > 0$, there exists $\delta > 0$ such that 
	$$ \abs*{\sqrt{x} - \L} < \epsilon_0 $$ 
for all $x$ such that $ 0 < \abs*{x - x_0} < \delta$.

\begin{align*}
	\abs*{\sqrt{x} - \L} &< \epsilon \\
	\abs*{\sqrt{x} - \L}\frac{\sqrt{x} + \L}{\sqrt{x} + \L} &< \epsilon \\
	\frac{\abs*{x - x_0}}{\sqrt{x} + \L} &< \epsilon
\end{align*}

We know that 
	$$ \frac{\abs*{x - x_0}}{\sqrt{x} + \L}  < \frac{\abs*{x - x_0}}{\L}$$
If we take  $ \abs*{x - x_0} < \delta = \epsilon \L $, then we get $$\frac{\abs*{x - x_0}}{\L} < \epsilon $$ 
So: 
	$$ \frac{\abs*{x - x_0}}{\sqrt{x} + \L} < \epsilon $$ 
Therefore our $\delta = \epsilon \L$ satisfies the conditions for the limit to exist (for $x_0 > 0$).

\end{proof}


\paragraph{Exercise 5.1.12.} Suppose that $x_0$ is a point of accumulation of both $A$ and $B$ and that $f:A \to \R$ and $g:B \to \R$.
We insist that $f$ and $g$ must agree in the sense that $f(x) = g(x)$ if $x$ is in both $A$ and $B$.
\begin{enumerate}[label=(\alph*)]
	\item What conditions on $A$ and $B$ ensure that if the limit $\limitToXZero f(x)$ exists so too must the limit $\limitToXZero g(x)$?

	Condition: There exists a $ \delta $, and $ I = (x_0 - \delta, x_0 + \delta) $, such that $$ B \cap I \subseteq A $$ and $ x_0 $ is an accumulation point of $ B \cap I $.
	\begin{proof}
		Within $ I $, every $b \in B $ must also be in $ A $. For every $ x $ that is in both $ A $ and $ B $, $ f(x) = g(x) $. By the sequential definition of a limit of a function, for all $ \set{\sn} $ where $ \sn \in A $ and $ \sn \to x_0$ as $ n \to \infty $, then $$ \limittoinfyn f(\sn) = \limitToXZero f(x)$$ 
		
		Furthermore, for any sequence $ \set{\tn} $ where $ \tn \in B \cap I$ and $ \tn \to x_0$ as $ n \to \infty $, then $$ \limittoinfyn g(\tn) = \limitToXZero g(x)$$
		However, since $B \cap I \subseteq A$, we know $ \set{\tn} $ will be a subsequence of some $ \set{\sn}$, therefore $$ \limittoinfyn g(\tn) = \limittoinfyn f(\sn) = \limitToXZero f(x)$$
		
		Therefore, $$ \limitToXZero f(x) = \limitToXZero g(x) $$
	\end{proof}

	\item What conditions on $A$ and $B$ ensure that if
    \[ \lim_{x \rightarrow x_0} f(x) \text{ and} \lim_{x \rightarrow x_0} g(x) \]
    both exist, then they must be equal?

	Condition: $ x_0 $ is a point of accumulation of $ A \cap B $
	\begin{proof}
		Since $ \limitToXZero f(x) $ and $ \limitToXZero g(x) $ exist, then for all $ \epsilon > 0 $, there exist $ \delta_1 > 0$, $\delta_2 >0 $ such that $$ |f(x) - L_1| < \epsilon $$ for all $x \in (x_0 - \delta_1, x_0 + \delta_1)$, and $$ |g(x) - L_2| < \epsilon $$ for all $x \in (x_0 - \delta_2, x_0 + \delta_2)$. 
		
		Let $ \delta = \min(\delta_1, \delta_2) $. From our condition, and by the definition of accumulation points, every interval $ (x - \delta, x + \delta) $ contains a point $x^*$ in $ A \cap B $. But for any point in this interval, $ |f(x^*) - L_1| < \epsilon $ and $ |g(x^*) - L_2| < \epsilon $, so:
			$$ |f(x^*) - L_1| + |g(x^*) - L_2| < \epsilon + \epsilon $$
		
		Furthermore, by the triangle inequality, 
		$$|f(x^*) - L_1 + L_2 - g(x^*)| \leq |f(x^*) - L_1| + |L_2 - g(x^*)|$$
		And because $ f(x^*) = g(x^*) $:
		$$|f(x^*) - L_1 + L_2 - g(x^*)| = |f(x^*) - L_1 + L_2 - f(x^*)| = | L_2 - L_1 | < 2\epsilon$$
		
		We can make $ 2\epsilon $ smaller than the distance between any two distinct numbers, therefore $$ L_2 = L_1 $$
	
					
%		It is given that $x_0$ is a accumulation point of both $ A $ and $ B $, there exists sequences $ \set{\sn} $ where $ \sn \in A $ and $ \sn \to x_0$ as $ n \to \infty $, and $ \set{\tn} $ where $ \tn \in B $ and $ \tn \to x_0$ as $ n \to \infty $. , then $$ \limittoinfyn g(\tn) = \limitToXZero g(x)$$
%		However, since $B \cap I \subseteq A$, we know $ \set{\tn} $ will be a subsequence of some $ \set{\sn}$, therefore $$ \limittoinfyn g(\tn) = \limittoinfyn f(\sn) = \limitToXZero f(x)$$
%		
%		 $$ \limittoinfyn f(\sn) = \limitToXZero f(x)$$ 
	\end{proof}

\end{enumerate}

\paragraph{Exercise 5.1.19.} Let $x_0$ be an accumulation point of the domain $E$ of a function $f$.
Prove that the limit $\lim_{x \rightarrow x_0} f(x)$ fails to exist if and only if there is a sequence of distinct points $\{e_n\}$ of $E$ converging to $x_0$ but with $ \set{ f(\en) } $ divergent.

\begin{proof}
%	TODO :verify claims
	
	We need to show that if $ \set{ f(\en)} $ diverges, then $\limittoinfyn f(\en)$ does not exist, and that if $\limittoinfyn f(\en)$ does not exist then $ \set{ f(\en)} $ diverges.
	
	% If $ \set{f(\en)}$ diverges then the limit does not exist.
	
	Assume that $$ \limitToXZero f(x) = L $$
	
	By definition of a function converging at $ x_0 $, for all sequences of $ \set{\en} $ such that $ \en \in E \quad \en \ne x_o $ and $ \en \to x_0 $ as $ n \to \infty $, then 
		$$ \limittoinfyn f(\en) = L$$
	But, $ \set{f(\en)} $ diverges, which means $ \limittoinfyn f(\en) $ does not exist, therefore $\limitToXZero f(x)$ also does not exist.
	
	Conversely, if $$ \limittoinfyn f(\en) $$ does not exist, then there exists a sequence $ \set{\en} $ where $ \en \in E $ with $ \en \ne x_0 $, and $ \en \to x_0 $ as $ n \to \infty $ such that $$ \limittoinfyn f(\en) $$ does not exist, so $ \set{ f(\en)} $ diverges.	
	
	Therefore the limit $\lim_{x \rightarrow x_0} f(x)$ fails to exist if and only if there is a sequence of distinct points $\{e_n\}$ of $E$ converging to $x_0$ but with $ \set{ f(\en) } $ divergent.

\end{proof}

\paragraph{Exercise 5.1.35.} Formulate a definition for the statements
\[ \lim_{x \rightarrow \infty} f(x) = \infty \text{ and} \lim_{x \rightarrow -\infty} f(x) = \infty. \]

\begin{enumerate}
	\item For all $ M > 0 $, there exists an $ x^* $ such that $ \forall x > x^* $, $ f(x) > M $.
	\item For all $ m < 0 $, there exists an $ x_* $ such that $ \forall x < x_* $, $ f(x) < m $.
\end{enumerate}


\section*{5.2 Properties of Limits}
\paragraph{Exercise 5.2.1.} Give an epsilon-delta proof of Theorem 5.10.
	
\begin{proof}
	Assume that 
		$$ \limitToXZero f(x_0) = L_1  \text{ and } \limitToXZero f(x_0) = L_2 $$
	where $ L_1 \ne L_2 $. By definition, there exist $ \epsilon_1 > 0 $ and $ \epsilon_2 > 0 $, such that
		$$ \abs{f(x_0) - L_1} < \epsilon_1  \text{ and } \abs{f(x_0) - L_2} < \epsilon_2 $$

	Since $ \epsilon_1 $ and $ \epsilon_2 $ can be arbitrarily small, then for all $\epsilon > 0$, we can choose $ \epsilon_1 $ and $ \epsilon_2 $ such that $\epsilon_1 + \epsilon_2 < \epsilon$.
	So, by algebra:
	\begin{align*}
		\abs{f(x_0) - L_1} + \abs{f(x_0) - L_2} &< \epsilon_1 + \epsilon_2 < \epsilon\\
		\abs{L_1 - f(x_0)} + \abs{f(x_0) - L_2} &< \epsilon \tag{Note: $ \abs{a} = \abs{-a} $}
	\end{align*}
	
	Furthermore, by the triangle inequality, 
	\begin{align*}
		\abs{L_1 - f(x_0) + f(x_0) - L_2} &\leq \abs{L_1 - f(x_0)} + \abs{f(x_0) - L_2} \\
		\abs{L_1 - L_2} &\leq \abs{L_1 - f(x_0)} + \abs{f(x_0) - L_2} \\
	\end{align*}

	Which means that $ \abs{L_1 - L_2} < \epsilon $. But we assumed that $L_1 \ne L_2$, so there is an $\epsilon < \abs{L_1 - L_2}$, which is a contradiction, therefore $L_1 = L_2$. The limit is unique.
	
\end{proof}


\newcommand{\dom}[1]{\text{dom}(#1)}
\newcommand{\inv}[1]{#1^{-1}}
\paragraph{Exercise 5.2.8.} Let $f$ and $g$ be functions with domains $\dom{f}$ and $\dom{g}$.
What are the domains of the functions listed below obtained by combining these functions algebraically or by a composition?
\begin{enumerate}[label=(\alph*)] %TODO
  \item $ \dom{f + g} = \dom{f} \cap \dom{g} $

  \item $ \dom{f - g} = \dom{f} \cap \dom{g} $

  \item $ \dom{fg} = \dom{f} \cap \dom{g} $

  \item $ \dom{f/g} = \dom{f} \cap \dom{g} \setminus \dom{\inv{g}(0)}$
  
  \item $ \dom{f \circ g} = \dom{g} \cap \inv{g}(\dom{f})$
  
  \item $ \dom{\sqrt{f + g}} = \dom{f} \cap \dom{g} \cap \set{x : f(x) + g(x) \ge 0} $
  
  \item $ \dom{\sqrt{fg}} = \dom{f} \cap \dom{g} \cap \set{x : f(x)g(x) \ge 0} $
  
\end{enumerate}

\paragraph{Exercise 5.2.11.} In the statement of Theorem 5.18 don't we also have to assume that $g(x)$ is never zero?

No. Limits do not care about the value of $ f(x_0) $, only the values in the neighborhood.

For example, take the functions:
\begin{align*}
	f(x) &= 1 \\
	g(x) &=
		\begin{cases}
		1 & \quad \text{if } x \ne 0 \\
		0 & \quad \text{if } x = 0
		\end{cases}
\end{align*}
Clearly, 
	$$\limitToXZero \frac{f(x)}{g(x)} = \frac{\limitToXZero f(x)}{\limitToXZero g(x)} = \frac{1}{1} = 1 $$
since $ \frac{f(x)}{g(x)} =1 $ in all neighborhoods of $ x_0 $, except at $x_0$, and 

\paragraph{Exercise 5.2.16.} Prove Theorem 5.17 by correcting the flawed epsilon-delta proof in Exercise 5.2.12 and by using the sequential definition of limit.
Which method is easier?

\textbf{Note to Reader}: The answer to this problem is not complete. The major flaw in the students proof is using 
	$$ \frac{\epsilon}{2|f(x)| + 1} \text{ instead of } \frac{\epsilon}{2|M| + 1} $$ but I have not worked out all the details. 

\begin{enumerate} 
	\item First off, the student uses, but does not define, L or M, so let 
		$$ \limitToXZero f(x) = L \text{ and } \limitToXZero g(x) = M$$
	By the definition of that $ \epsilon > 0 $.
	Choose $ \delta_1 $ so that
		$$ |f(x) - L| <\frac{\epsilon}{2|M| + 1} $$
	if $ 0 < |x - x_0| < \delta_1 $ and also choose $ \delta_2 $ so that 
		$$ |g(x) - M| < \frac{\epsilon}{2|L| + 1} $$ 
	if $ 0 < |x - x_0| < \delta_2 $. 
	
	Define $ \delta = \min \set{\delta_1, \delta_2} $. If $ 0 < |x - x_0| < \delta $, then we have 
	$$ |f(x)g(x) - LM| \leq |f(x)| |g(x) - M| + |M| |f(x) - L| \leq≤ |f(x)| \frac{\epsilon}{2|f(x)| + 1} + |M| \frac{\epsilon}{2|M| + 1} < \epsilon $$. 
	
	Therefore $ f(x)g(x) \to LM $ if $ f(x) \to L $ and $ g(x) \to M $.”
	
	\item 
	Let $ \set{\sn}$ and be any sequence of points in $ \dom{f} $ such that $ \sn \to x_0 $ as $ n \to \infty $, and let $ \set{\tn}$ and be any sequence of points in $ \dom{g} $ such that $ \tn \to x_0 $, as $ n \to \infty $. By the theorem for products of limits of sequences, we know that $$ \limittoinfyn (\sn\tn)  = \left(\limittoinfyn \sn \right) \left(\limittoinfyn \tn \right)$$
\end{enumerate}


\end{document}
