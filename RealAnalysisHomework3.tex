\documentclass[11pt]{article}
% decent example of doing mathematics and proofs in LaTeX.
% An Incredible degree of information can be found at
% http://en.wikibooks.org/wiki/LaTeX/Mathematics

% Use wide margins, but not quite so wide as fullpage.sty
\marginparwidth 0.25in 
\oddsidemargin 0.25in 
\evensidemargin 0.25in 
\marginparsep 0.25in
\topmargin 0.25in 
\textwidth 6in \textheight 8 in

\usepackage[utf8]{inputenc}
\usepackage[english]{babel}
\usepackage{amsthm}
\usepackage{amsmath}
\usepackage{amssymb}
\usepackage{amsfonts}
\usepackage{upgreek}
\usepackage{semantic}
\usepackage{enumerate}

\begin{document}
	\author{Paul Wintz}
	\title{Homework Set 3aasdf}
	\maketitle
	
	\newcommand{\limittoinfy}[2]{\lim_{#1 \to \infty}#2}
	\newcommand{\limittoinfyn}[1]{\lim_{n \to \infty}#1}
	\def \limsup {\lim_{n \to \infty}sup}
	\def \liminf {\lim_{n \to \infty}inf}
	\newcommand{\sumto}[1]{\sum_{n=1}^{#1}}
	\def \sumtoinfy {\sum_{n=1}^{\infty}}

	\newcommand{\set}[1]{\{#1\}}
	\def \sn {s_n}
	\def \tn {t_n}
	\def \xn {x_n}
	\def \an {a_n}
	\def \bn {b_n}
	
	\def \snplusone{s_{n+1}}
	\def \snfrac {\frac{\snplusone}{\sn}}
	
	\def \nfrac{\frac{1}{n}}
	\def \novernplusone {\frac{n}{n+1}}
	
	\section*{2.8.7}
	\subsection*{Suppose that $\set{\sn}$ and $\set{\tn}$ are sequences of positive numbers, that $$\limittoinfyn{\frac{\sn}{\tn}} = \infty$$ and that $\tn \rightarrow \infty$ What can you conclude?}
	
	By the quotient theorem for limits, we know that $$\limittoinfyn{\frac{\sn}{\tn}} = \frac{\limittoinfyn{\sn}}{\limittoinfyn{\tn}} = \infty$$
	Then, by the definition of divergence to infinity
	$$\forall M \in \mathbb{R}, \frac{\limittoinfyn{\sn}}{\limittoinfyn{\tn}} >M$$
	By algebra of limits, we get
	$$\limittoinfyn{\sn} > M \limittoinfyn{\tn}$$	
	Since M can be any large, positive number, then for $M > 1$
	$$\limittoinfyn{\sn} > M \limittoinfyn{\tn} > \limittoinfyn{\tn}$$
	Since $\tn \rightarrow \infty$, we can conclude that $\sn \rightarrow \infty$.
	
	\section*{2.8.9}
	\subsection*{
		Let $\set{\sn}$ be a sequence of positive numbers. Show that the condition $$\limittoinfyn{\snfrac{}} < 1$$ implies $\sn \rightarrow 0$.}
	
		\begin{align*}
			\limittoinfyn{\snfrac{}} = L &< 1 \\
			\text{Choose a number $\alpha$ such that: } \\
			\limittoinfyn{\snfrac{}} < \alpha &< 1 \\
			\exists N \text{ such that $\forall n > N$} \\
			\limittoinfyn{\snfrac{}} < \alpha &< 1 \\	
			x_n < \alpha^n 
			\text{As $n \rightarrow \infty$ then $\alpha^n \rightarrow 0$, therefore: } \\
			\xn \rightarrow 0 \\					
		\end{align*}
	
	\section*{2.9.6} 
	\subsection*{Prove the nested interval property: if given a sequence of closed intervals $$[a_1, b_1] \supset [a_2, b_2] \supset [a_3, b_3] \supset \dots $$ arranged so that each interval is a subinterval of the one preceding it and so that the lengths of the intervals shrink to zero, then there is exactly one point that belongs to every interval of the sequence. }
	
	
	\begin{proof} 
		$\\a_n \leq b_1 \implies a_n$ has an upper bound. \\
		$b_n \geq a_1 \implies b_n$ has a lower bound. \\
		$[a_{n+1}, b_{n+1}]$ is a subinterval of $[a_n, b_n]$, so $a_n$ is nondecreasing monotonically, and $b_n$ is nonincreasing monotonically. \\
		
		\begin{align*} %Theorem 2.28 (Monotone Convergence Theorem)			
		|b_n - a_n| \rightarrow 0 \tag{Given} \\
		|b_n - a_n | = (b_n - a_n) \tag{because $b_n \geq a_n$} \\
		\implies (b_n - a_n) < \epsilon,\ \forall \epsilon > 0 \tag{By definition of convergence}\\
		a_n \leq b_n \\ 
		\limittoinfyn{a_n} \leq \limittoinfyn{b_n} \tag{Theorem 2.18} \\
		\text{Assume: } \limittoinfyn{a_n} = A \ne B = \limittoinfyn{b_n} \\
		A, B \in \bigcap _{{\bigcup }_{\cdot }=1}^{\infty} [aj, bj] \\
		\text{We can choose an epsilon such that} \\
		b_n - a_n < \epsilon < B - A \\
		b_n - a_n < B - A \\
		\implies \text{both A or B cannot both be in the interval} \\
		\end{align*}
		Which is a contradiction, therefore $A = B$ and $\limittoinfyn{\an} = \limittoinfyn{\bn}$
	\end{proof}
	
	\subsection*{Would it be true for a descending sequence of open intervals $$(a_1, b_1) \supset (a_2, b_2) \supset (a_3, b_3) \supset \dots $$}
		Counter example: for the open interval (0, $\nfrac$), the  "because $\nfrac, 0 < \epsilon$"
	
	% TODO 
	\section*{2.10.3} \subsection*{Define $e = \limittoinfyn{(1 + 1/n)^n}$. Show that $2 < e < 3$}
	
	\def \nfactorfrac {\frac{1}{n!}}
	\begin{proof} $\newline$		
		From the text, it has been shown that: $$e = \limittoinfyn{(1 + \nfrac)^n} = \sum_{n=0}^{\infty} \nfactorfrac $$

		We can define an partial estimation of e:
		$$e_n = \sum_{i=0}^{n} s_i$$ where $s_i = \frac{1}{i!}$
		
		Because $\nfactorfrac$ is always positive, we now that $e_k$ is strictly increasing. Therefore, since $e_1 = \frac{1}{0!} + \frac{1}{1!} = 2$, we know that $e > 2$.

		We can show $e < 3$ by finding a series, $\sum_{n=0}^{\infty}\an$ such that $\an \geq e_n\ \forall n$, strictly greater for some values, and that converges to 3. 
		We define $a_1 = 1, a_n = \frac{1}{2}^{n-1}$ is such a sequence. The sum of this series will be $1 + \sum_{n=1}^{\infty}\frac{1}{2}^{n-1} = 1 + 1 + 1/2 + 1/4 + \dots = 1 + \frac{1}{1-1/2} = 3$.  
		The first three terms of both $\an$ and $s_n$ are $\set{1, 1, 1/2}$. For n=3
		$$a_3 = 1 / 2^{3-1} = 1/4 < s_3 = 1/3! = 1/6s$$. 
		For n+1, $a_{n+1} = \frac{1}{2}a_n$ and $s_{n+1} = \frac{1}{n+1}s_n$. For $n\geq3, \frac{1}{n+1} < \frac{1}{2}$, and we have shown that $s_n < a_n$ for $n=3$, so by induction $s_n < a_n \forall n \geq 3$
		Therefore the value of $e < \sum_{n=0}^{\infty}\an = 3$.
	\end{proof}
	
	\section*{2.11.6} 
	\subsection*{Establish which of the following statements are true.}
	\begin{enumerate}[(a)]
		\item \textbf{A sequence is convergent if and only if all of its subsequences are convergent.} \\ 
		True. A sequence is a subsequence of itself, therefore if a sequence is divergent, not all  subsequence is not convergent
		\item \textbf{A sequence is bounded if and only if all of its subsequences are bounded.} \\
		True
		\item \textbf{A sequence is monotonic if and only if all of its subsequences are monotonic.} \\
		True
		\item \textbf{A sequence is divergent if and only if all of its subsequences are divergent.} \\
		False
	\end{enumerate}
	
	\def \theseq {\{\xn\}}
	\def \subeleone {x_{m_k}}
	\def \subeletwo {x_{n_k}}
	\def \subseqone {\{\subeleone\}}
	\def \subseqtwo {\{x_{n_k}\}}
	\def \startofsubseq {x_{n_0}}
	
	\section*{2.11.28}
	\subsection*{Let $\set{x_n}$ be a bounded sequence that diverges. Show that there is a pair of convergent subsequences $\set{x_{n_k}}$ and $\set{x_{m_k}}$ so that $\limittoinfy{k}{|x_{n_k} - x_{m_k}|} > 0$}
	
	% Theorem 2.40 (Bolzano-Weierstrass): Every bounded sequence contains a convergent subsequence.
	\begin{proof}
				
		Since the sequence is bounded but does not converge, then by the Monotonic Convergence Theorem, it must not be monotonic. Furthermore, if any tail of the sequence converges, then the sequence must converge, so none of the tails converge and must also be non-monotonic. Since all tails are non-monotonic, then there exists a value M, such that $\forall \startofsubseq, n_0 > M$, there will be an element later in the sequence that is greater than $\startofsubseq$, and another element that is less than the $\startofsubseq$. We can define a subsequence, $\subseqone$, that starts with $\startofsubseq$ then for skips to the next element of $\theseq$ that is larger than the last element in $\subseqone$. Since $\subseqone$ is increasing monotonically, then by the Monotonic Convergence Theorem, it must converge to a value, $L_1$ that is larger than $\startofsubseq$. Similarly, we can create a monotonically decreasing subsequence, $\subseqtwo$, starting at $\startofsubseq$ that must converge to a limit, $L_2$ that is strictly less than $\startofsubseq$. 
				 
		 We have shown that $L_1$ is greater than $\startofsubseq$ and $L_2 < \startofsubseq$, so  $L_1 \ne L_2$ and $|L_1 - L_2| > 0$, therefore:
		 	 $$|L_1 - L_2| = |\limittoinfy{k}{\subeleone} - \limittoinfy{k}{\subeletwo}| = \limittoinfy{k}{|\subeleone - \subeletwo|} > 0$$
				
	\end{proof}

	\section*{2.12.10}
	
	\section*{2.13.15} 
	\subsection*{Let S denote the set of all real numbers t with the property that some subsequence of a given sequence $\set{an}$ converges to t. What is the relation between the set S and the lim sups and lim infs of the sequence $\set{an}$?}
	
	%If $sup(S) > $ 
	
	\section*{2.14.2} 
	\subsection*{For any convergent sequence$\set{\an}$ write $\sn = \frac{a_1 + a_2 + \dots + \an}{n}$, the sequence of averages. Show that $\limittoinfyn{\an} = \limittoinfyn{\sn}.$}
	
	
	\subsection*{Give an example to show that $\set{\sn}$ could converge even if $\set{\an}$ diverges.}
	
	\section*{2.14.6}
	\subsection*{For any convergent sequence $\set{\an}$ write $\sn = \sqrt[n]{a_1a_2 \dots a_n}$, the sequence of geometric averages. Show that $\limittoinfyn{a_n} = \limittoinfyn{\sn}$.}
	
	\subsection*{Give an example to show that	$\set{\sn}$ could converge even if $\set{\an}$ diverges.}
	
	
	
	
\end{document}
